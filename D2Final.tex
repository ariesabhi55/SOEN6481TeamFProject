
\documentclass[paper=a4, fontsize=11pt]{report}

\usepackage[utf8]{inputenc}
\usepackage{titlesec}

\usepackage[numberedsection]{glossaries}

\makeglossaries
 
\newglossaryentry{User Stories}
{
        name=User Stories,
        description={are informal, natural language description of one or more features of a software system}
}
\newglossaryentry{Traceability Matrix}
{
        name=Traceability Matrix,
        description={is a document, usually in the form of a table, used to assist in determining the completeness of a relationship by correlating any two base-lined documents using a many-to-many relationship comparison}
}
 

%%new packages

\usepackage{amsmath}
\usepackage{latexsym}
\usepackage{amsfonts}
\usepackage[normalem]{ulem}
\usepackage{array}
\usepackage{amssymb}
\usepackage{graphicx}
\usepackage[backend=biber,
style=numeric,
sorting=none,
isbn=false,
doi=false,
url=false,
]{biblatex}\addbibresource{bibliography.bib}

\usepackage{subfig}
\usepackage{wrapfig}
\usepackage{wasysym}
\usepackage{enumitem}
\usepackage{adjustbox}
\usepackage{ragged2e}
\usepackage[svgnames,table]{xcolor}
\usepackage{tikz}
\usepackage{longtable}
\usepackage{changepage}
\usepackage{setspace}
\usepackage{hhline}
\usepackage{multicol}
\usepackage{tabto}
\usepackage{float}
\usepackage{multirow}
\usepackage{makecell}
\usepackage{fancyhdr}
\usepackage[toc,page]{appendix}
\usepackage[hidelinks]{hyperref}
\usetikzlibrary{shapes.symbols,shapes.geometric,shadows,arrows.meta}
\tikzset{>={Latex[width=1.5mm,length=2mm]}}
\usepackage{flowchart}\usepackage[paperheight=11.0in,paperwidth=8.5in,left=1in,right=1in,top=0.5in,bottom=0.5in,headheight=1in]{geometry}
\usepackage[utf8]{inputenc}
\usepackage[T1]{fontenc}
\TabPositions{0.5in,1.0in,1.5in,2.0in,2.5in,3.0in,3.5in,4.0in,4.5in,5.0in,5.5in,6.0in,6.5in,7.0in,}

\urlstyle{same}


 %%%%%%%%%%%%  Set Depths for Sections  %%%%%%%%%%%%%%

% 1) Section
% 1.1) SubSection
% 1.1.1) SubSubSection
% 1.1.1.1) Paragraph
% 1.1.1.1.1) Subparagraph


\setcounter{tocdepth}{5}
\setcounter{secnumdepth}{5}


 %%%%%%%%%%%%  Set Depths for Nested Lists created by \begin{enumerate}  %%%%%%%%%%%%%%


\setlistdepth{9}
\renewlist{enumerate}{enumerate}{9}
		\setlist[enumerate,1]{label=\arabic*)}
		\setlist[enumerate,2]{label=\alph*)}
		\setlist[enumerate,3]{label=(\roman*)}
		\setlist[enumerate,4]{label=(\arabic*)}
		\setlist[enumerate,5]{label=(\Alph*)}
		\setlist[enumerate,6]{label=(\Roman*)}
		\setlist[enumerate,7]{label=\arabic*}
		\setlist[enumerate,8]{label=\alph*}
		\setlist[enumerate,9]{label=\roman*}

\renewlist{itemize}{itemize}{9}
		\setlist[itemize]{label=$\cdot$}
		\setlist[itemize,1]{label=\textbullet}
		\setlist[itemize,2]{label=$\circ$}
		\setlist[itemize,3]{label=$\ast$}
		\setlist[itemize,4]{label=$\dagger$}
		\setlist[itemize,5]{label=$\triangleright$}
		\setlist[itemize,6]{label=$\bigstar$}
		\setlist[itemize,7]{label=$\blacklozenge$}
		\setlist[itemize,8]{label=$\prime$}

\setlength{\topsep}{0pt}\setlength{\parskip}{8.04pt}
\setlength{\parindent}{0pt}

\renewcommand{\arraystretch}{1.3}

%%New packages

\usepackage[T1]{fontenc}
\usepackage{fourier}

\usepackage{caption}

\usepackage[english]{babel}															% English language/hyphenation
\usepackage[protrusion=true,expansion=true]{microtype}	
\usepackage{amsmath,amsfonts,amsthm} % Math packages

%%% Custom sectioning
\usepackage{sectsty}
\allsectionsfont{\centering \normalfont\scshape}



%%% Equation and float numbering
\numberwithin{equation}{section}		% Equationnumbering: section.eq#
\numberwithin{figure}{section}			% Figurenumbering: section.fig#
\numberwithin{table}{section}				% Tablenumbering: section.tab#


%%% Maketitle metadata
\newcommand{\horrule}[1]{\rule{\linewidth}{#1}} 	% Horizontal rule

\titleformat{\chapter}{\centering\normalfont\huge}{\thechapter.}{20pt}{\huge}

%%% Begin document
\begin{document}

\begin{titlepage}
    \centering
    \vfill
    \includegraphics[width=9cm]{Concordia-University-logo.png}
    {\bfseries\Large
        \vskip2cm
        Department of Computer Science and Software Engineering \\
        \vskip2cm
        SOEN 6481: SOFTWARE SYSTEMS REQUIREMENTS SPECIFICATION\\
        \vskip2cm
        Abhishek Rajput\\
        Student ID: 40093879\\
        \vskip9mm
        22 July, 2019\\
    }    
    \vfill
    %\includegraphics[width=4cm]{Concordia-University-logo.png} 
    \vfill
    \vfill
\end{titlepage}

\tableofcontents

\chapter{Abstract}
This project investigates and describes the concept of Champernowne Constant C\textsubscript{10} which is a transcendental real constant. For base 10, the number is defined by concatenating representations of successive integers i.e. 0.12345678910111213141516...
\vskip1mm
Since it's decimal expansion has important and unique properties, many mathematicians and teachers have found this number profoundly useful. It is constructed in such a way that it's decimal digits are easy to investigate. This allows establishing easily that it is normal in its base.
\vskip1mm
During the project, I have also interviewed two people with strong mathematical background regarding this constant and its application. Moreover, I have created a persona based on the analysis of the interview that was conducted.
\vskip1mm
In this report, I have discussed concepts relevant to Calculator for Champernowne Constant. It includes a description of each concept, relationships between concepts and a Domain Model.
\vskip1mm
Additionally, illustration with the description of each Use Case, Use Case Diagram and Activity Diagram for the Use Cases and UML for the normal scenario of each use case has also been included in this report.


\chapter{Acknowledgement}
I would like to express my deepest appreciation to all those who helped me and provided me the possibility to complete this project.
\vskip1mm
This project would not have been possible without the essential and gracious support of Prof. P. Kamthan whose contribution in stimulating suggestions and encouragement guided me throughout.
\vskip1mm
I would also like to express my sincere gratitude to our Teaching Assistant, Mr. Mehran Ishanian, who had given his assistance to clarify my doubts during this project.
\vskip1mm
Furthermore, I would also like to acknowledge with much appreciation the crucial role of the interviewee Dr. Pankaj Srivastava and Mr. Aayush Sharma, who gave their valuable time from their busy schedule to help me in completion of this project. 
\vskip1mm
Finally, I would like to thank my family and friends for all their understanding and support. 


\chapter{Changes made in the Deliverable 1 (D1)}
For problem 5 (Use Case Model)[Chapter 7, Page no. 12]- Identifiers for use cases have been changed from 1,2,3.. to UC1, UC2, UC3.. respectively so that they can be referenced accordingly in a traceability matrix.


%%User Stories starts

\chapter{\gls{User Stories} (Problem 6)}
In this section, I will cover user stories relevant to Calculator for Champernowne Constant. \newline

  
\section{User Stories relevant to calculator for Champernowne Constant }

%%user stories starts

\begin{center}
\begin{tabular}{| m{15cm} |} 
\hline
\textbf{\large US1} \\ [0.7ex]
\hline\hline
As a user, I want to generate the Champernowne Constant. \\
\hline
\textbf{Constraint} \newline Generated number must be a decimal number.  \\ 
\hline
\textbf{Priority} \newline 5 \\
\hline
\textbf{Estimate} \newline 13 \\
\hline
\textbf{Acceptance Tests} \newline a). Must be in base 10.\newline
b). Must have a decimal point at second place.\newline
c). Complete number must be readable.\newline
d). Number must get displayed within 10 seconds. \\
\hline
\end{tabular}
\end{center}
\vspace{1.5em}
%%%2
\begin{center}
\begin{tabular}{| m{15cm} |} 
\hline
\textbf{\large US2} \\ [0.7ex]
\hline\hline
As a user, I want to choose the number of digits after the decimal point in generated Champernowne Constant. \\
\hline
\textbf{Constraint} \newline Chosen number of digits must be a positive integer.  \\ 
\hline
\textbf{Priority} \newline 4 \\
\hline
\textbf{Estimate} \newline 5 \\
\hline
\textbf{Acceptance Tests} \newline a). Number of digits entered must be integer only. \\
\hline
\end{tabular}
\end{center}
\vspace{1.5em}
%%3
\begin{center}
\begin{tabular}{| m{15cm} |} 
\hline
\textbf{\large US3} \\ [0.7ex]
\hline\hline
As a user, I want to calculate the number of occurrences of a particular number in generated Champernowne Constant. \\
\hline
\textbf{Constraint} \newline Desired number must be a positive integer.  \\ 
\hline
\textbf{Priority} \newline 4 \\
\hline
\textbf{Estimate} \newline 13 \\
\hline
\textbf{Acceptance Tests} \newline a). Occurrences obtained must be for the entered number only.\newline
b). The result must get displayed within 10 seconds. \\
\hline
\end{tabular}
\end{center}
\vspace{1.5em}

%%4
\begin{center}
\begin{tabular}{| m{15cm} |} 
\hline
\textbf{\large US4} \\ [0.7ex]
\hline\hline
As a user, I want to calculate the position of the first occurrence of a particular number in generated Champernowne Constant. \\
\hline
\textbf{Constraint} \newline Calculated place must be for first occurrence.  \\
\hline
\textbf{Priority} \newline 4 \\
\hline
\textbf{Estimate} \newline 13 \\
\hline
\textbf{Acceptance Tests} \newline a). Position obtained must be for the entered number only.\newline
b). The position displayed must be of the first occurrence only.\newline
c). The result must get displayed within 10 seconds. \\
\hline
\end{tabular}
\end{center}
\vspace{1.5em}

%%5
\begin{center}
\begin{tabular}{| m{15cm} |} 
\hline
\textbf{\large US5} \\ [0.7ex]
\hline\hline
As a user, I want to store the result, to use it at a later time, if needed. \\
\hline
\textbf{Constraint} \newline Stored number must be reusable.  \\ 
\hline
\textbf{Priority} \newline 2 \\
\hline
\textbf{Estimate} \newline 3 \\
\hline
\textbf{Acceptance Tests} \newline a). User must get the confirmation regarding the operation. \\
\hline
\end{tabular}
\end{center}
\vspace{1.5em}
%%6
\begin{center}
\begin{tabular}{| m{15cm} |} 
\hline
\textbf{\large US6} \\ [0.7ex]
\hline\hline
As a user, I want to retrieve the stored result, whenever needed. \\
\hline
\textbf{Constraint} \newline Retrieved result be same as stored result.  \\ 
\hline
\textbf{Priority} \newline 2 \\
\hline
\textbf{Estimate} \newline 3 \\
\hline
\textbf{Acceptance Tests} \newline a). User must get the stored result only. \\
\hline
\end{tabular}
\end{center}

\vspace{1.5em}
%%7
\begin{center}
\begin{tabular}{| m{15cm} |} 
\hline
\textbf{\large US7} \\ [0.7ex]
\hline\hline
As a user, I want to delete the content of the input field one by one, from end. \\
\hline
\textbf{Constraint} \newline Only one value must be deleted at a time.  \\ 
\hline
\textbf{Priority} \newline 3 \\
\hline
\textbf{Estimate} \newline 1 \\
\hline
\textbf{Acceptance Tests} \newline a). Only one input must get deleted at a time. \\
\hline
\end{tabular}
\end{center}
\vspace{1.5em}

%%8
\begin{center}
\begin{tabular}{| m{15cm} |} 
\hline
\textbf{\large US8} \\ [0.7ex]
\hline\hline
As a user, I want to clear the content of the input field. \\
\hline
\textbf{Constraint} \newline Complete input field must get cleared at once.  \\ 
\hline
\textbf{Priority} \newline 3 \\
\hline
\textbf{Estimate} \newline 1 \\
\hline
\textbf{Acceptance Tests} \newline a). Input field must get cleared at once. \\
\hline
\end{tabular}
\end{center}
\vspace{1.5em}
%%9
\begin{center}
\begin{tabular}{| m{15cm} |} 
\hline
\textbf{\large US9} \\ [0.7ex]
\hline\hline
As a user, I want to clear the content of the output screen. \\
\hline
\textbf{Constraint} \newline Complete output screen must get cleared at once.  \\ 
\hline
\textbf{Priority} \newline 2 \\
\hline
\textbf{Estimate} \newline 1 \\
\hline
\textbf{Acceptance Tests} \newline a). Output screen must get cleared at once. \\
\hline
\end{tabular}
\end{center}
\vspace{1.5em}
%%10
\begin{center}
\begin{tabular}{| m{15cm} |} 
\hline
\textbf{\large US10} \\ [0.7ex]
\hline\hline
As a user, I want to do basic calculations: Addition, Subtraction, Multiplication, and Division. \\
\hline
\textbf{Constraint} \newline Calculations must be correct. \\ 
\hline
\textbf{Priority} \newline 4 \\
\hline
\textbf{Estimate} \newline 8 \\
\hline
\textbf{Acceptance Tests} \newline a). Results must be correct.\newline
b). The result must get displayed within 10 seconds. \\
\hline
\end{tabular}
\end{center}
\vspace{1.5em}
%%11
\begin{center}
\begin{tabular}{| m{15cm} |} 
\hline
\textbf{\large US11}\\ [0.7ex]
\hline\hline
As a user, I want to insert values in input field, either using calculator keys or using a keyboard or both. \\
\hline
\textbf{Constraint} \newline Both input ways must produce same results.  \\ 
\hline
\textbf{Priority} \newline 5 \\
\hline
\textbf{Estimate} \newline 8 \\
\hline
\textbf{Acceptance Tests} \newline a). User must be able to provide inputs using both keyboard and keypad available on the screen. \\
\hline
\end{tabular}
\end{center}

%%User stories ends

%% traceability matrix starts

\chapter{Backward \gls{Traceability Matrix} (Problem 7)}
In this section, I will cover backward traceability matrix relevant to Calculator for Champernowne Constant. \newline

  
\section{Backward Traceability Matrix relevant to calculator for Champernowne Constant }

\begin{center}
\begin{tabular}{| m{.8cm} | m{1.8cm} | m{.8cm} | m{.8cm}| m{3cm} | m{1cm} | m{3cm} | m{1.5cm} |} 
\hline
S. No. & User Stories & UC & US & Interview & Survey & Global & Persona \\ [0.7ex]
\hline\hline
1. & US1 & UC1 &  &  &  &  & \\ 
\hline
2 & US2 &  &  & Question: 16, 18 &  &  &   \\ 
\hline
3 & US3 & UC4 &  & Question: 13 &  &  &  \\ 
\hline
4 & US4 & UC5 &  & Question: 13 &  &  &  \\ 
\hline
5 & US5 & UC8 &  &  &  &  &  \\ 
\hline
6 & US6 & UC9 & US5 &  &  &  &  \\ 
\hline
7 & US7 & UC6 &  &  &  & Windows Calculator &  \\ 
\hline
8 & US8 & UC7 &  &  &  & Windows Calculator & \\ 
\hline
9 & US9 &  &  &  &  & Windows Calculator &  \\ 
\hline
10 & US10 & UC2 &  &  &  & Windows Calculator &  \\ 
\hline
11 & US11 &  &  &  &  & Windows Calculator &  \\ 
\hline
\end{tabular}
\end{center}

%% traceability matrix ends


%%Glossary section starts
\printglossary


\chapter{Project Workspace Address}
\textbf{GitHub Project Workspace Address:}
\url{https://github.com/ariesabhi55/SOEN6481TeamFProject} 

%%References section starts
\chapter{References}

\url{https://en.wikipedia.org/wiki/Champernowne\_constant}
\vskip1mm
\url{http://mathworld.wolfram.com/ChampernowneConstant.html}
\vskip1mm
\url{https://en.wikipedia.org/wiki/User\_story}
%%% End document
\end{document}