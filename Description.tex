
\documentclass[paper=a4, fontsize=11pt]{scrartcl}
\usepackage[T1]{fontenc}
\usepackage{fourier}

\usepackage[english]{babel}															% English language/hyphenation
\usepackage[protrusion=true,expansion=true]{microtype}	
\usepackage{amsmath,amsfonts,amsthm} % Math packages
\usepackage[pdftex]{graphicx}	
\usepackage{url}


%%% Custom sectioning
\usepackage{sectsty}
\allsectionsfont{\centering \normalfont\scshape}


%%% Custom headers/footers (fancyhdr package)
\usepackage{fancyhdr}
\pagestyle{fancyplain}
\fancyhead{}											% No page header
\fancyfoot[L]{}											% Empty 
\fancyfoot[C]{}											% Empty
\fancyfoot[R]{\thepage}									% Pagenumbering
\renewcommand{\headrulewidth}{0pt}			% Remove header underlines
\renewcommand{\footrulewidth}{0pt}				% Remove footer underlines
\setlength{\headheight}{13.6pt}


%%% Equation and float numbering
\numberwithin{equation}{section}		% Equationnumbering: section.eq#
\numberwithin{figure}{section}			% Figurenumbering: section.fig#
\numberwithin{table}{section}				% Tablenumbering: section.tab#


%%% Maketitle metadata
\newcommand{\horrule}[1]{\rule{\linewidth}{#1}} 	% Horizontal rule

\title{
		%\vspace{-1in} 	
		\usefont{OT1}{bch}{b}{n}
		\normalfont \normalsize \textsc{Department of Computer Science and Software Engineering } \\ [25pt]
		\horrule{0.5pt} \\[0.4cm]
		\huge Champernowne Constant (C10) \\
		\horrule{2pt} \\[0.5cm]
}

\author{Abhishek Rajput\\ \texttt{aries.abhi55@gmail.com}} % Author name and email address
\usepackage{datetime}
\newdate{date}{05}{07}{2019}
\date{\displaydate{date}}

%%% Begin document
\begin{document}
\maketitle
\section*{Introduction}
In mathematics, the Champernowne constant C10 is a transcendental real constant whose decimal expansion has important properties. It is named after economist and mathematician D. G. Champernowne, who published it as an undergraduate in 1933.
For base 10, the number is defined by concatenating representations of successive integers:
\[ C10 = 0.12345678910111213141516\cdots \]  

Champernowne constants can also be constructed in other bases, similarly, for example:

\[ C2 = 0.11011100101110111\cdots \]
\[ C3 = 0.12101112202122\cdots \]

The Champernowne constants can be expressed exactly as infinite series:\newline

{\displaystyle C_{m}=\sum _{n=1}^{\infty }{\frac {n}{10_{b}^{~\left(\sum \limits _{k=1}^{n}\left\lceil \log _{10_{b}}(k+1)\right\rceil \right)}}}} \newline



where {\displaystyle \lceil {x}\rceil =} {\displaystyle \lceil {x}\rceil =} ceiling( {\displaystyle x} ), {\displaystyle 10_{b}^{~x}=b^{x}} {\displaystyle 10_{b}^{~x}=b^{x}} in base 10, {\displaystyle \log _{10_{b}}(x)=\log _{b_{10}}(x)} {\displaystyle \log _{10_{b}}(x)=\log _{b_{10}}(x)} \newline and {\displaystyle b}\ is\ the\ base\ of\ the\ constant. \newline

A slightly different expression is given by Eric W. Weisstein (MathWorld): \newline

{\displaystyle C_{m}=\sum _{n=1}^{\infty }{\frac {n}{m^{\left(n+\sum \limits _{k=1}^{n-1}\left\lfloor \log _{m}(k+1)\right\rfloor \right)}}}}
 where {\displaystyle \lfloor {x}\rfloor =} {\displaystyle \lfloor {x}\rfloor =} floor( {\displaystyle x} x).
\newline
\section*{References}
https://en.wikipedia.org/wiki/Champernowne\_constant
%%% End document
\end{document}