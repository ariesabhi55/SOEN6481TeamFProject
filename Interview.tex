
\documentclass[paper=a4, fontsize=11pt]{scrartcl}
\usepackage[T1]{fontenc}
\usepackage{fourier}

\usepackage[english]{babel}															% English language/hyphenation
\usepackage[protrusion=true,expansion=true]{microtype}	
\usepackage{amsmath,amsfonts,amsthm} % Math packages
\usepackage[pdftex]{graphicx}	
\usepackage{url}


%%% Custom sectioning
\usepackage{sectsty}
\allsectionsfont{\centering \normalfont\scshape}


%%% Custom headers/footers (fancyhdr package)
\usepackage{fancyhdr}
\pagestyle{fancyplain}
\fancyhead{}											% No page header
\fancyfoot[L]{}											% Empty 
\fancyfoot[C]{}											% Empty
\fancyfoot[R]{\thepage}									% Pagenumbering
\renewcommand{\headrulewidth}{0pt}			% Remove header underlines
\renewcommand{\footrulewidth}{0pt}				% Remove footer underlines
\setlength{\headheight}{13.6pt}


%%% Equation and float numbering
\numberwithin{equation}{section}		% Equationnumbering: section.eq#
\numberwithin{figure}{section}			% Figurenumbering: section.fig#
\numberwithin{table}{section}				% Tablenumbering: section.tab#


%%% Maketitle metadata
\newcommand{\horrule}[1]{\rule{\linewidth}{#1}} 	% Horizontal rule

\title{
		%\vspace{-1in} 	
		\usefont{OT1}{bch}{b}{n}
		\normalfont \normalsize \textsc{Department of Computer Science and Software Engineering } \\ [25pt]
		\horrule{0.5pt} \\[0.4cm]
		\huge Champernowne Constant (C10) \\
		\horrule{2pt} \\[0.5cm]
}

\author{Abhishek Rajput\\ \texttt{aries.abhi55@gmail.com}} % Author name and email address
\usepackage{datetime}
\newdate{date}{11}{07}{2019}
\date{\displaydate{date}}

%%% Begin document
\begin{document}
\maketitle
\section*{Introduction}
In this document, I will discuss the interview conducted with the potential users of Champernowne Constant. \newline
It includes criteria for the selection of interviewees, questions asked in the interview, interviewee's responses to the questions and my analysis of this interview process.

\section*{Criteria used to select Interviewee}
As Champernowne Constant is famous for its transcendental and normal nature so in my opinion, a person having a strong background in mathematics would be the most suitable match for being an interviewee.
Keeping this thing in mind, I interviewed below two people over the phone based on my designed questionnaire.

\begin{flushleft}
\textbf{1. Interviewee 1}
\\Name: Dr. Pankaj Srivastava 
\\Qualification: M.Sc Mathematics, Ph.D
\end{flushleft}

\begin{flushleft}
\textbf{2. Interviewee 2}
\\Name: Aayush Sharma 
\\Qualification: M.Sc Mathematics
\end{flushleft}

\section*{Interview Questions \& Responses}
\begin{flushleft}
\textbf{Question 1: How would you define Champernowne Constant?}
\\Response:
\\Interviewee 1: A number having a sequence of positive integers after a decimal.
\\Interviewee 2: Simply a decimal following with positive integers numbers till any number of places.
\end{flushleft}

\begin{flushleft}
\setlength{\parskip}{\baselineskip}
\hrulefill
\\\textbf{Question 2: What is the first thing that comes to your mind when you hear about Champernowne Constant?}
\\Response:
\\Interviewee 1: Number having transcendental and normal number properties
\\Interviewee 2: Normal number
\end{flushleft}

\begin{flushleft}
\setlength{\parskip}{\baselineskip}
\hrulefill
\\\textbf{Question 3: How is this normal number different from other normal numbers?}
\\Response:
\\Interviewee 1: As far as I know, because this number is normal in both base 2 and 10.
\\Interviewee 2: Well, not very sure about that.
\end{flushleft}

\begin{flushleft}
\setlength{\parskip}{\baselineskip}
\hrulefill
\\\textbf{Question 4: Have you ever used this number in your career?}
\\Response:
\\Interviewee 1: Not many times, but yes, sometimes to give an example of transcendental and normal number to my students.
\\Interviewee 2: Never used, but read about it.
\end{flushleft}

\begin{flushleft}
\setlength{\parskip}{\baselineskip}
\hrulefill
\\\textbf{Question 5: Do you know of any application which uses this number? }
\\Response:
\\Interviewee 1: I read that this number can trick any program which attempts to find regularity in sequences.
\\Interviewee 2: No
\end{flushleft}

\begin{flushleft}
\setlength{\parskip}{\baselineskip}
\hrulefill
\\\textbf{Question 6: Do you know of any field in which it could prove its worth?}
\\Response:
\\Interviewee 1: In my teaching field for sure.
\\Interviewee 2: In the education field, to show what are transcendental and normal numbers.
\end{flushleft}

\begin{flushleft}
\setlength{\parskip}{\baselineskip}
\hrulefill
\\\textbf{Question 7: Can we use this number in combination with any other number to perform some useful tasks?}
\\Response:
\\Interviewee 1: As we get this number after concatenating positive integers after a decimal point  so if we multiply this number with 1 having leading 0s equal to digits after the decimal places then 
we can eventually get a number having digits as the sequence of positive integers.
\\Interviewee 2: No idea.
\end{flushleft}

\begin{flushleft}
\setlength{\parskip}{\baselineskip}
\hrulefill
\\\textbf{Question 8: If I build a calculator to calculate this number what features would you like to be present in that calculator?}
\\Response:
\\Interviewee 1: As this number is normal, your calculator should provide the functionality to find the occurrences of a certain number in this constant and should provide other similar kinds of functionalities as well.
\\Interviewee 2: To get this number up-to any limit.
\end{flushleft}

\begin{flushleft}
\setlength{\parskip}{\baselineskip}
\hrulefill
\\\textbf{Question 9: Could you please tell me any other feature that can make this calculator more useful?}
\\Response:
\\Interviewee 1: You could probably add the functionality to generate graphs of the results obtained. As you know, graphs give a better visualization of results.
\\Interviewee 2: If possible, compute other complex numbers too.
\end{flushleft}

\begin{flushleft}
\setlength{\parskip}{\baselineskip}
\hrulefill
\\\textbf{Question 10: How can this calculator be useful to you?}
\\Response:
\\Interviewee 1: If I can prove to my students that why this number is a normal number, it would definitely be useful for me.
\\Interviewee 2: To generate this number of any length.
\end{flushleft}

\begin{flushleft}
\setlength{\parskip}{\baselineskip}
\hrulefill
\\\textbf{Question 11: How many digits after the decimal should be generated by the calculator for this number?}
\\Response:
\\Interviewee 1: You should give the user the flexibility to enter the precision level.
\\Interviewee 2: I think it should be user dependent.
\end{flushleft}

\begin{flushleft}
\setlength{\parskip}{\baselineskip}
\hrulefill
\\\textbf{Question 12: In that calculator, would you like to get the result in different formats?}
\\Response:
\\Interviewee 1: I don't think there is any need for that.
\\Interviewee 2: Why not, if possible.
\end{flushleft}

\begin{flushleft}
\setlength{\parskip}{\baselineskip}
\hrulefill
\\\textbf{Question 13: What inputs would you want to give to the calculator in order to generate this number?}
\\Response:
\\Interviewee 1: I think just the precision level should be enough.
\\Interviewee 2: Base and number of digits after the decimal.
\end{flushleft}

\begin{flushleft}
\setlength{\parskip}{\baselineskip}
\hrulefill
\\\textbf{Question 14: How would you like to use this calculator (Console or UI)?}
\\Response:
\\Interviewee 1: Either way, but a calculator with UI would be easier to use.
\\Interviewee 2: UI, for sure.
\end{flushleft}

\begin{flushleft}
\setlength{\parskip}{\baselineskip}
\hrulefill
\\\textbf{Question 15: If I choose to build a calculator with UI, what things should be kept in mind to have a good UX?}
\\Response:
\\Interviewee 1: Good placement of buttons.
\\Interviewee 2: Number should be displayed properly, proper alignment of components, good color contrast.
\end{flushleft}

\section*{Conclusion}
\begin{flushleft}
From this interview, I have concluded below points:

1. This number is mostly popular for its Normal behavior.
\\2. This number can trick programs which attempts to find regularity in sequences.
\\3. The number of places to display after the decimal in the calculator should be dependent on the user's input.
\\4. It is better to have UI for this calculator.
\\5. For additional functionality, you can add 'Generate Graph Functionality'.
\\6. We can use this number to find the position of any number in the string made of a sequence of positive integers.
\\7. We can also use this number to answer questions like- How many digits are there in total in the two digit numbers 10 to 99, Find all the positions at which a string occurs, Compute the position at which a given number (string of digits) first appears in the sequence, number of occurrences of certain number etc.

\end{flushleft}


%%% End document
\end{document}