
\documentclass[paper=a4, fontsize=11pt]{scrartcl}
\usepackage[T1]{fontenc}
\usepackage{fourier}

\usepackage[english]{babel}															% English language/hyphenation
\usepackage[protrusion=true,expansion=true]{microtype}	
\usepackage{amsmath,amsfonts,amsthm} % Math packages
\usepackage[pdftex]{graphicx}	
\usepackage{url}


%%% Custom sectioning
\usepackage{sectsty}
\allsectionsfont{\centering \normalfont\scshape}


%%% Custom headers/footers (fancyhdr package)
\usepackage{fancyhdr}
\pagestyle{fancyplain}
\fancyhead{}											% No page header
\fancyfoot[L]{}											% Empty 
\fancyfoot[C]{}											% Empty
\fancyfoot[R]{\thepage}									% Pagenumbering
\renewcommand{\headrulewidth}{0pt}			% Remove header underlines
\renewcommand{\footrulewidth}{0pt}				% Remove footer underlines
\setlength{\headheight}{13.6pt}


%%% Equation and float numbering
\numberwithin{equation}{section}		% Equationnumbering: section.eq#
\numberwithin{figure}{section}			% Figurenumbering: section.fig#
\numberwithin{table}{section}				% Tablenumbering: section.tab#


%%% Maketitle metadata
\newcommand{\horrule}[1]{\rule{\linewidth}{#1}} 	% Horizontal rule

\title{
		%\vspace{-1in} 	
		\usefont{OT1}{bch}{b}{n}
		\normalfont \normalsize \textsc{Department of Computer Science and Software Engineering } \\ [25pt]
		\horrule{0.5pt} \\[0.4cm]
		\huge Champernowne Constant (C10) \\
		\horrule{2pt} \\[0.5cm]
}

\author{Abhishek Rajput\\ \texttt{aries.abhi55@gmail.com}} % Author name and email address
\usepackage{datetime}
\newdate{date}{05}{07}{2019}
\date{\displaydate{date}}

%%% Begin document
\begin{document}
\maketitle
\section*{Introduction}
In mathematics, the Champernowne constant C10, named after economist and mathematician D. G. Champernowne, is a transcendental real constant whose decimal expansion has important properties.\newline
Transcendental numbers are the numbers which are not the root of any polynomial with integer coeffients, i.e., opposite of algebraic numbers.\newline
For base 10, the number is defined by concatenating representations of successive integers:
\[ C10 = 0.12345678910111213141516\cdots \]  

Champernowne constants can also be constructed in other bases, similarly, for example:

\[ C2 = 0.11011100101110111\cdots \]
\[ C3 = 0.12101112202122\cdots \]

The Champernowne constants can be expressed exactly as infinite series:

\begin{align*}
{\displaystyle C_{m}=\sum _{n=1}^{\infty }{\frac {n}{10_{b}^{~\left(\sum \limits _{k=1}^{n}\left\lceil \log _{10_{b}}(k+1)\right\rceil \right)}}}}
\end{align*}

\begin{align*}
where\ {\displaystyle \lceil {x}\rceil =} \ ceiling(x), {\displaystyle 10_{b}^{~x}=b^{x}}\ in\ base\ 10,\ {\displaystyle \log _{10_{b}}(x)=\log _{b_{10}}(x)} \ and\  b\ is\ the\ base\ of \ the \ constant.
\end{align*}

\begin{flushleft}
In simpler words, we can say that Champernowne Constant is formed by taking the sequence of whole numbers, i.e., 1, 2, 3, 4, 5, and so on and putting them behind a decimal point.\newline

This constant is interesting because it can be understood to contain an encoding of any past, present or future information, because any given sequence of numbers can be shown to exist somewhere in the champernowne representation.\newline

Another interesting feature of this number that if a person is picking up a number then there is 1/10 chance of getting a specific one digit number, 1/100 chance of getting a specific two digit number, 1/1000 chance of getting a specific three digit number and so on.\newline

In other words, for Champernowne Constant, in base 10, we would expect strings [0],[1],[2],...,[9] to occur 1/10 of the time, strings [0,0],[0,1],...,[9,8],[9,9] to occur 1/100 of the time, and so on, which also implies that Champernowne Constant is normal in base 10.
\end{flushleft}

\section*{References}
https://en.wikipedia.org/wiki/Champernowne\_constant
%%% End document
\end{document}
